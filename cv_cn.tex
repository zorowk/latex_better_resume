%------------------------------------
% Dario Taraborelli
% Typesetting your academic CV in LaTeX
%
% URL: http://nitens.org/taraborelli/cvtex
% DISCLAIMER: This template is provided for free and without any guarantee 
% that it will correctly compile on your system if you have a non-standard  
% configuration.
% Some rights reserved: http://creativecommons.org/licenses/by-sa/3.0/
%------------------------------------

%!TEX TS-program = xelatex
%!TEX encoding = UTF-8 Unicode

\documentclass[10pt, a4paper]{article}
\usepackage{fontspec} 

% DOCUMENT LAYOUT
\usepackage{geometry} 
\geometry{a4paper, textwidth=5.5in, textheight=8.5in, marginparsep=7pt, marginparwidth=.6in}
\setlength\parindent{0in}

% FONTS
\usepackage[usenames,dvipsnames]{xcolor}
\usepackage{xunicode}
\usepackage{xltxtra}
\defaultfontfeatures{Mapping=tex-text}
%\setromanfont [Ligatures={Common}, Numbers={OldStyle}, Variant=01]{Linux Libertine O}
%\setmonofont[Scale=0.8]{Monaco}
%%% modified by Karol Kozioł for ShareLaTeX use
\setmainfont[
  Ligatures={Common}, Numbers={OldStyle}, Variant=01,
  BoldFont=LinLibertine_RB.otf,
  ItalicFont=LinLibertine_RI.otf,
  BoldItalicFont=LinLibertine_RBI.otf
]{LinLibertine_R.otf}
\setmonofont[Scale=0.8]{DejaVuSansMono.ttf}

\usepackage[CJKchecksingle]{xeCJK}
\setCJKmainfont[
    BoldFont={Noto Sans CJK SC Bold}
    ]{Noto Sans CJK SC}
\setCJKsansfont{Noto Sans CJK SC}
\setCJKmonofont{Migu 1M}
\punctstyle{hangmobanjiao}

% ---- CUSTOM COMMANDS
\chardef\&="E050
\newcommand{\html}[1]{\href{#1}{\scriptsize\textsc{[html]}}}
\newcommand{\pdf}[1]{\href{#1}{\scriptsize\textsc{[pdf]}}}
\newcommand{\doi}[1]{\href{#1}{\scriptsize\textsc{[doi]}}}
% ---- MARGIN YEARS
\usepackage{marginnote}
\newcommand{\amper{}}{\chardef\amper="E0BD }
\newcommand{\years}[1]{\marginnote{\scriptsize #1}}
\renewcommand*{\raggedleftmarginnote}{}
\setlength{\marginparsep}{7pt}
\reversemarginpar

% HEADINGS
\usepackage{sectsty} 
\usepackage[normalem]{ulem} 
\sectionfont{\mdseries\upshape\Large}
\subsectionfont{\mdseries\scshape\normalsize} 
\subsubsectionfont{\mdseries\upshape\large} 

% PDF SETUP
% ---- FILL IN HERE THE DOC TITLE AND AUTHOR
\usepackage[%dvipdfm, 
bookmarks, colorlinks, breaklinks, unicode,
% ---- FILL IN HERE THE TITLE AND AUTHOR
	pdftitle={C++软件工程师},
	pdfauthor={王阳明},
  pdfproducer={xelatex}
]{hyperref}  
\hypersetup{linkcolor=blue,citecolor=blue,filecolor=black,urlcolor=MidnightBlue} 
 
\pagestyle{empty}  % no page number for the second and the later pages  
\thispagestyle{empty} % no page number for the first page

% DOCUMENT
\begin{document}
{\LARGE 王阳明}\\[1cm]

 Born:   Juny 02, 1994-大同市, 山西\\
 Phone: \texttt{168-3427-5659}\\
 email: \href{mailto:personalemail@outlook.com}{personalemail@outlook.com}\\

%\hrule
\section*{教育经历}
\noindent
\years{2011-2015}\textsc{英国国王学院}, 计算机科学\\

%%\hrule
\section*{工作经历} 
\years{2017-2018}XXXXX科技有限公司, \emph{项目经理}\\
\years{2016-2017}XXXXX软件开发公司, \emph{软件工程师}\\
\years{2015-2016}XXXXX科技有限公司, \emph{软件工程师}\\

\section*{项目经验}

\subsection*{\years{2017-2018} \textbf{视频播放器}}
\noindent
\small{项目的摘要介绍,我在此项目负责了哪些工作,分别在哪些地方做得出色/和别人不一样/成长快,这个项目中,我最困难的问题是什么,我采取了什么措施,最后结果如何。这个项目中,我最自豪的技术细节是什么,为什么,实施前和实施后的数据对比如何,同事和领导对此的反应如何}
\begin{itemize}
  \item[-] 在项目中使用boost库通过查找资料完成视频解码器模块
  \item[-] 改进对于网络的支持,研究RTSP/RTP/RTCP协议族,完成服务器在终端的播放
  \item[-] 优化视频播放的网络数据传输部分,保证在低宽带下的视频流畅播放
\end{itemize}

\subsection*{\years{2016-2017} \textbf{透明加解密}}
\noindent
\small{项目的摘要介绍,我在此项目负责了哪些工作,分别在哪些地方做得出色/和别人不一样/成长快,这个项目中,我最困难的问题是什么,我采取了什么措施,最后结果如何。这个项目中,我最自豪的技术细节是什么,为什么,实施前和实施后的数据对比如何,同事和领导对此的反应如何}
\begin{itemize}
  \item[-] 对比研究常用加密算法的性能以及安全性,选择使用aes加密算法
  \item[-] 完成对于系统的hook引擎的设计,主要用来劫持系统的复制,拖放,以及文件
    读取部分,自动在读取和存放的时候加入加密和解密过程
  \item[-] 优化hook引擎
\end{itemize}
\subsection*{\years{2012-2014} \textbf{ABCD项目}}
\noindent
\small{项目的摘要介绍,我在此项目负责了哪些工作,分别在哪些地方做得出色/和别人不一样/成长快,这个项目中,我最困难的问题是什么,我采取了什么措施,最后结果如何。这个项目中,我最自豪的技术细节是什么,为什么,实施前和实施后的数据对比如何,同事和领导对此的反应如何}
\begin{itemize}
  \item[-] 使用什么技术完成什么模块,对于该模块自豪的地方
  \item[-] 使用什么技术完成什么模块,以及该模块值得自豪的数据信息
  \item[-] 使用什么技术完成什么模块
\end{itemize}

\nopagebreak[4]

%\hrule
\section*{工作技能}
\noindent
\begin{itemize}
  \item[-] 熟悉C++ 11,Qt 库4年使用,boost 库2年使用
  \item[-] 熟练Linux操作系统
  \item[-] 熟悉python,java编写过服务器后台,熟悉它们和C++的通信过程  
\end{itemize}

\begin{center}
{\scriptsize  Last updated: \today\- •\- 
% ---- PLEASE LEAVE THIS BACKLINK FOR ATTRIBUTION AS PER CC-LICENSE
Typeset in \href{http://nitens.org/taraborelli/cvtex}{
%\fontspec{Times New Roman}
\XeTeX }\\
% ---- FILL IN THE FULL URL TO YOUR CV HERE
% \href{http://nitens.org/taraborelli/cvtex}{http://nitens.org/taraborelli/cvtex}
}
\end{center}

\end{document}
